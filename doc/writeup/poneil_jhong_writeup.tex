%
% File acl2012.tex
%
% Contact: Maggie Li (cswjli@comp.polyu.edu.hk), Michael White (mwhite@ling.osu.edu)
%%
%% Based on the style files for ACL2008 by Joakim Nivre and Noah Smith
%% and that of ACL2010 by Jing-Shin Chang and Philipp Koehn


\documentclass[11pt]{article}
\usepackage{poneil_jhong}
\usepackage{times}
\usepackage{latexsym}
\usepackage{amsmath}
\usepackage{multirow}
\usepackage{url}
\DeclareMathOperator*{\argmax}{arg\,max}
\setlength\titlebox{6.5cm}    % Expanding the titlebox

\title{Meta Dynamic Programming}

\author{Juneki Hong \\
  Johns Hopkins University \\
  {\tt juneki@jhu.edu} \\\And
  Paul O'Neil \\
  Johns Hopkins University \\
  {\tt poneil1@jhu.edu} \\}

\date{}

\begin{document}
\maketitle
\begin{abstract}
  We built a server in order to perform multiple exact local alignment problems. This server manages a storage database and worker nodes in order to handle queries made by clients. As clients send many local alignment problem requests, we want to be able to store and cache previous solutions for use later. The client-server architecture will provide a framework for this task. 
\end{abstract}


\section{Introduction}
%Why work on this? Why your approach?
The task of local alignment is a frequently queried task in Computational Genomics. 
An individual local alignment problem can be solved by computing a matrix of values using Dynamic Programming, but what if parts of a problem matrix could be reused in future alignment tasks?
When many queries for local alignment are made, there might be some similar structure across all of the tasks that could be taken advantage of. For example, when aligning gene sequences, similar sequences might reoccur because they were all taken from a single group of microorganisms, or even simply because of the limited permutations of the alphabet A,C,G,T. 

We chose to build a server in order to recieve alignment requests, store past queries, and then return the result. This server can serve as a central location for previous alignment tasks, where we could try to fetch a solution from a database, or solve the problem when none were available. With this server in place, it could be possible to explore different ways we can exploit the similarities across our problems.
 



\section{Prior Work}
% What did you read? What did others accomplish before you?


\section{Methods and software}
%What did you implement? Why? How?

\subsection{Architecture}
The architecture of our project divides into four logical sections: A Client, a Leader, Storage, and Workers.
% the why. The how.

\subsubsection{Client}
The Client uploads genomes to the Leader, and then sends queries for Local Alignment problems. The Leader will handle these queries and respond with Local Alignment solutions in the form of a completed matrix and the location of the maximum value.

\subsubsection{Leader}
The Leader handles queries from the Client as well as allocates work to Workers and manage the solutions. A Leader can allocate work to a Worker by specifying the border initialization, that is the first row and column to which it can begin filling out the rest of the matrix. A Worker can complete a solution and inform the Leader of its location in Storage. Once a solution has been indicated, a Leader can then query Storage to pick it up, to send back to the Client. 

\subsubsection{Storage}
The Storage is a database for the Leader and Worker to store and query items such as genomes, and solutions to alignment problems. 
A Worker will produce solutions and store them into Storage. It will also check to see if a solution already exists, which then it does not have to do any additional work.

\subsubsection{Worker}
A Worker queries the Leader whenever it is available, asking for problems to work on. The Leader can then respond with problems that are available which the Worker can claim. A Worker claims a problem specification, of which it can begin filling out a matrix and complete a solution, storing the Solution in Storage.
If a solution already exists in Storage, then a Worker does not have to do additional work and just tell the Leader its location.


\subsection{Problem Subdivision}

A feature we wanted our Leader to have was problem subdivision.

When we implemented this feature, we introduced bugs that interfered with the operation of the Leader and Storage, so we had to remove it. The Leader curently sends entire problem specifications to the Workers instead of problem chunks whose solutions are reassembled by the Leader.


\subsection{Partial solution matching}

Another feature we wanted Storage to have was partial solution matching. Given a problem specification, which...


\section{Results}

How well did your method work compared to others?

\section{Conclusions} 

What did you learn? What should we come away with?


\section{Works Cited}



\begin{thebibliography}{}

\bibitem[\protect\citename{Aho and Ullman}1972]{Aho:72}
Alfred~V. Aho and Jeffrey~D. Ullman.
\newblock 1972.
\newblock {\em The Theory of Parsing, Translation and Compiling}, volume~1.
\newblock Prentice-{Hall}, Englewood Cliffs, NJ.

\bibitem[\protect\citename{{American Psychological Association}}1983]{APA:83}
{American Psychological Association}.
\newblock 1983.
\newblock {\em Publications Manual}.
\newblock American Psychological Association, Washington, DC.

\bibitem[\protect\citename{{Association for Computing Machinery}}1983]{ACM:83}
{Association for Computing Machinery}.
\newblock 1983.
\newblock {\em Computing Reviews}, 24(11):503--512.

\bibitem[\protect\citename{Chandra \bgroup et al.\egroup }1981]{Chandra:81}
Ashok~K. Chandra, Dexter~C. Kozen, and Larry~J. Stockmeyer.
\newblock 1981.
\newblock Alternation.
\newblock {\em Journal of the Association for Computing Machinery},
  28(1):114--133.

\bibitem[\protect\citename{Gusfield}1997]{Gusfield:97}
Dan Gusfield.
\newblock 1997.
\newblock {\em Algorithms on Strings, Trees and Sequences}.
\newblock Cambridge University Press, Cambridge, UK.

\end{thebibliography}

\end{document}
