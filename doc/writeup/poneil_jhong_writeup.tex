%
% File acl2012.tex
%
% Contact: Maggie Li (cswjli@comp.polyu.edu.hk), Michael White (mwhite@ling.osu.edu)
%%
%% Based on the style files for ACL2008 by Joakim Nivre and Noah Smith
%% and that of ACL2010 by Jing-Shin Chang and Philipp Koehn


\documentclass[11pt]{article}
\usepackage{poneil_jhong}
\usepackage{times}
\usepackage{latexsym}
\usepackage{amsmath}
\usepackage{multirow}
\usepackage{url}
\DeclareMathOperator*{\argmax}{arg\,max}
\setlength\titlebox{6.5cm}    % Expanding the titlebox

\title{Meta Dynamic Programming}

\author{Juneki Hong \\
  Johns Hopkins University \\
  {\tt juneki@jhu.edu} \\\And
  Paul O'Neil \\
  Johns Hopkins University \\
  {\tt poneil1@jhu.edu} \\}

\date{}

\begin{document}
\maketitle
\begin{abstract}
  We built a server in order to perform multiple exact local alignment problems. This server manages a storage database and worker nodes in order to handle queries made by clients. As clients send many local alignment problem requests, we want to be able to store and cache previous solutions for use later. The client-server architecture will provide a framework for this task. 
\end{abstract}


\section{Introduction}
Why work on this? Why your approach?

The task of local alignment is a frequently queried task in Computational Genomics. When many queries are made, there might be some similar structure across all of the tasks that could be taken advantage of. For example, when aligning gene sequences, similar sequences might reoccur because they were all taken from a single group of microorganisms, or even simply because of the limited permutations of the alphabet A,C,G,T. 

We chose to build a server in order to recieve alignment requests, store past queries, and then return the result. The server could then try to fetching a solution from a database, or solve the problem when none were available.
 



\section{Prior Work}
 What did you read? What did others accomplish before you?


\section{Methods and software}

What did you implement? Why? How?

\subsection{Architecture}
The 

\subsubsection{Client}

\subsubsection{Leader}

\subsubsection{Storage}

\subsubsection{Worker}



\subsection{Problem Subdivision}

A feature we wanted our Leader

\subsection{Partial solution matching}


\section{Results}

How well did your method work compared to others?

\section{Conclusions} 

What did you learn? What should we come away with?


\section{Works Cited}



\begin{thebibliography}{}

\bibitem[\protect\citename{Aho and Ullman}1972]{Aho:72}
Alfred~V. Aho and Jeffrey~D. Ullman.
\newblock 1972.
\newblock {\em The Theory of Parsing, Translation and Compiling}, volume~1.
\newblock Prentice-{Hall}, Englewood Cliffs, NJ.

\bibitem[\protect\citename{{American Psychological Association}}1983]{APA:83}
{American Psychological Association}.
\newblock 1983.
\newblock {\em Publications Manual}.
\newblock American Psychological Association, Washington, DC.

\bibitem[\protect\citename{{Association for Computing Machinery}}1983]{ACM:83}
{Association for Computing Machinery}.
\newblock 1983.
\newblock {\em Computing Reviews}, 24(11):503--512.

\bibitem[\protect\citename{Chandra \bgroup et al.\egroup }1981]{Chandra:81}
Ashok~K. Chandra, Dexter~C. Kozen, and Larry~J. Stockmeyer.
\newblock 1981.
\newblock Alternation.
\newblock {\em Journal of the Association for Computing Machinery},
  28(1):114--133.

\bibitem[\protect\citename{Gusfield}1997]{Gusfield:97}
Dan Gusfield.
\newblock 1997.
\newblock {\em Algorithms on Strings, Trees and Sequences}.
\newblock Cambridge University Press, Cambridge, UK.

\end{thebibliography}

\end{document}
